\documentclass[a4paper]{article}
\usepackage{fancyhdr}
\usepackage[pdftex]{graphicx}
\usepackage{sidecap}
\usepackage{listings}
\usepackage{color}
\usepackage[export]{adjustbox}
\usepackage{subcaption}
\usepackage{graphicx}
\usepackage{filecontents}
\usepackage{pgffor}
\usepackage{forloop}
\usepackage{amsmath}
\usepackage{amssymb}
\usepackage{lineno}
\usepackage{amsmath}
\usepackage{hyperref}
\usepackage{mathtools}
\usepackage{amsmath,amsthm,amssymb}
\numberwithin{equation}{subsection}
\numberwithin{figure}{subsection}
\definecolor{codegreen}{rgb}{0,0.6,0}
\definecolor{codegray}{rgb}{0.5,0.5,0.5}
\definecolor{codepurple}{rgb}{0.58,0,0.82}
\definecolor{backcolour}{rgb}{0.95,0.95,0.92}
\definecolor{mygreen}{RGB}{25,172,0} % color values Red, Green, Blue
\definecolor{mylilas}{RGB}{170,55,241}
\definecolor{dkgreen}{rgb}{0,0.6,0}
\definecolor{gray}{rgb}{0.5,0.5,0.5}
\definecolor{mauve}{rgb}{0.58,0,0.82}

\usepackage{hyperref}
\hypersetup{
    colorlinks=true,
    linkcolor=blue,
    filecolor=magenta, 
    urlcolor=cyan,
    pdfpagemode=FullScreen,
}
\usepackage{geometry}
 \geometry{
 a4paper,
 total={210mm,297mm},
 left=8mm,
 right=8mm,
 top=15mm,
 bottom=15mm,
 }

\usepackage{glossaries}


\makeglossaries
\newglossaryentry{FSM}
{
    name=FSM,
    description={Finite State Machine}
}



\pagestyle{fancy}
\fancyhf{}
\rhead{\href{Vangjush.Komini@uzleuven.be}{Vangjush Komini}}
\lhead{\href{https://www.kuleuven.be/kuleuven/}{KU Leuven}}
\rfoot{Page \thepage}
\lfoot{Biomedical Data Processing part 2}


\lstset{inputpath=Code1}
\graphicspath{{Images1/}}

\include{Glossary}


\begin{titlepage}

\title{Assignment1\\\centerline{\textit{Subspace Signal Processing}} }

\author{
\href{mailto:vangjush.komini@uzleuven.be}{Vangjush Komini}\\  \textit{r0612470} \\
\href{mailto:vangjush.komini@uzleuven.be}{vangjush.komini@uzleuven.be}\\
}





\end{titlepage}




\begin{document}



\maketitle
\begin{center}
\Large \href{https://onderwijsaanbod.kuleuven.be/syllabi/e/H06W1AE.htm#activetab=doelstellingen_idp41200}{Biomedical Data Processing Part II}
\end{center}

\begin{figure}[!htbp]
\centering
\includegraphics[width=0.3\textwidth]{icon2.png}
\end{figure}

\newpage
\section{Introduction}

The underlying methodology of subspace signal processing is the decomposition of the signal into mutually orthogonal subspaces wherein one of them consist of the noisy subspace. This is ensured under the assumption that the signal itself contains two parts. One is the low-rank linear model for the "clean" signal and the additive uncorrelated (white) noise. Therefore the energy of the least correlated noise contaminating the whole signal is concentrated in a lower subspace compare to the highly correlated signal. Recovering the wanted signal after decomposition is performed by mapping the highly correlated subspaces onto the space where the structure is the closest to the clean signal. Additionally, the noise is supposed to be white before application of this decomposition, otherwise, whitening of the noise is required in case it is colored. This is possible with the noise covariance matrix to be known. Apart from filtering, subspace signal processing could also be employed for parameter estimation of an underlying estimation model. The outcome is significantly less biased compared to other existing methodologies, though they suffer from serious drawback. Very little prior knowledge is possible to incorporate into the framework. Last but not least, subspace signal processing enables simultaneous processing of multiple channels. This is an important feature since it is possible to benefit from cross-over information. Magnetic resonance spectroscopy MRS is one of the domains where subspace signals are heavily utilized. The general outline of the subspace signal processing is as follow:

\begin{itemize}
    \item separate the $(\hat{H}=>signal+noise)$  subspace where the original signal $H=\hat{H}+W$
    \item suppress the $(W=>noise-only)$ subspaces from the noise only subspace. 
    \item further suppression of the noise in the reconstructed space. 
\end{itemize}



\section{CPD}


\begin{figure}[!htbp]
\minipage{0.5\textwidth}%
\centering
\includegraphics[width=1\linewidth]{400.jpg}
\subcaption{Covariance matrix (2x2)}
\endminipage\hfill
\minipage{0.5\textwidth}%
\centering
\includegraphics[width=1\linewidth]{401.jpg}
\subcaption{Fourth order cumulative tensor (3x3x3x3) stretched at (9x9).}
\endminipage\hfill
\caption{Tensors overview}
\end{figure}

Their rank is estimated to be two for both cases and the number of dimensions (order) for the first and the second tensor are respectively two and four. The CPD relation for this will be as equation \ref{C2} and \ref{C4} respectively to C2 and C4 dataset.

\begin{equation}\label{C2}
    C2=\sum_{r=1}^{2} U_{r}^{(1)}\otimes U_{r}^{(1)}
\end{equation}

\begin{equation}\label{C4}
    C4=\sum_{r=1}^{2} U_{r}^{(1)}\otimes U_{r}^{(2)}\otimes U_{r}^{(3)}\otimes U_{r}^{(4)}}
\end{equation}

Alternatively this could be re-written into a rang $(L_{r},L_{r},1)$ of a block term decomposition as in the equation \ref{BTD}

\begin{equation}\label{BTD}
    Tensor=A(C\Theta B)^{T}
\end{equation}

where $\Theta$ is Khatri–Rao product of tensors. This is a suitable notation for alternative least square algorithm. The component A,B and C component for tensor C2 and C4 are respectively in figure \ref{C22} and \ref{C44}. 

\begin{figure}[!htbp]
\minipage{0.33\textwidth}%
\centering
\includegraphics[width=1\linewidth]{402.jpg}
\subcaption{A component.}
\endminipage\hfill
\minipage{0.33\textwidth}%
\centering
\includegraphics[width=1\linewidth]{403.jpg}
\subcaption{B component}
\endminipage\hfill
\minipage{0.33\textwidth}%
\centering
\includegraphics[width=1\linewidth]{404.jpg}
\subcaption{C component}
\endminipage\hfill
\caption{CPD decomposition of C2 matrix}\label{C22}
\end{figure}




\begin{figure}[!htbp]
\minipage{0.33\textwidth}%
\centering
\includegraphics[width=1\linewidth]{405.jpg}
\subcaption{A component.}
\endminipage\hfill
\minipage{0.33\textwidth}%
\centering
\includegraphics[width=1\linewidth]{406.jpg}
\subcaption{B component.}
\endminipage\hfill
\minipage{0.33\textwidth}%
\centering
\includegraphics[width=1\linewidth]{407.jpg}
\subcaption{C component.}
\endminipage\hfill
\caption{CPD decomposition of C4 tensor}\label{C44}
\end{figure}


Tensorlab toolbox has also been utilized to do the rank estimation and the CPD decomposition. The matrices U from equation \ref{C2} and \ref{C4} are plotted in figures \ref{C222} and \ref{C444} respectively.



\begin{figure}[!htbp]
\minipage{0.5\textwidth}%
\centering
\includegraphics[width=1\linewidth]{408.jpg}
\subcaption{U1 component.}
\endminipage\hfill
\minipage{0.5\textwidth}%
\centering
\includegraphics[width=1\linewidth]{409.jpg}
\subcaption{U2 component.}
\endminipage\hfill
\caption{CPD decomposition of C2 matrix}\label{C222}
\end{figure}


\begin{figure}[!htbp]
\minipage{0.5\textwidth}%
\centering
\includegraphics[width=1\linewidth]{410.jpg}
\subcaption{U1 component.}
\includegraphics[width=1\linewidth]{412.jpg}
\subcaption{U3 component.}
\endminipage\hfill
\minipage{0.5\textwidth}%
\centering
\includegraphics[width=1\linewidth]{411.jpg}
\subcaption{U2 component.}
\includegraphics[width=1\linewidth]{413.jpg}
\subcaption{U4 component.}
\endminipage\hfill
\caption{CPD decomposition of C4 matrix}\label{C444}
\end{figure}

\section{Parameter estimation}\label{sec2}

In order to apply subspace methods for parameter estimation, an underlying model has to be utilized with well defined parameters. Exponentially damped sinusoidal $(EDS)$ is a very convenient model and widely utilized in different signal processing applications figure \ref{EDS}. Hereby the signal being detected in the sensing device is modeled as a superposition of damping signals at distinct frequencies contaminated with white Gaussian noise $(wgn)$ equation \ref{eq1}. These components accommodate the information regarding the tissue properties, therefore accurate estimation of the amplitude $a_{k}$, frequency $f_{k}$, phase $\phi_{k}$ and damping $d_{k}$ is very critical. The analysis is performed in the frequency domain.   

 
\begin{equation}\label{eq1}
y(t)=\sum_{k=1}^{K} a_{k}exp(j\phi_{k})exp(-d_{k}t+2\pi f_{k}t)\delta t+wgn
\end{equation}

Utilizing the estimated parameter, a modeled signal (EDS) is then employed for peaks of interest computation . Via subspace methods based on (total) least square approximation a best fit between the modeled and the original signal is computed. This will enable a good estimation of peaks closely spaced,as well as their amplitudes. 

Herein the main algorithm being introduced is outlined in \ref{Ap2} which fits the model either via least square (LS) or via total least square (TLS). Additionally prior knowledge of poles are also being incorporated in the algorithm for further increase of the accuracy of the estimation. In order to validate the robustness of the methods, wgn is superimposed to the water filtered signal and the estimated parameters are then compared to the wgn free signal. 


Contrary to LS solution which introduces higher bias and inconsistency, TLS methods introduce an asymptotically bias compare to the ground truth solution \cite{7}. The known poles will enable the orthogonal projection of the data via QR decomposition whereby it will remove the known part \ref{Ap3}. 

In figure \ref{Nadya1} are the spectra of the reconstruction signal from the noise free signal respectively to the three methods. Whereas on the figure \ref{Nadya2} there are the spectra of the reconstructed signal from the noisy signal for respective method. 

The performance of each method is outlined in the table \ref{Nadya3}. The first row corresponds to the absolute mean value of the difference between the original and the fitted signal $\Delta=Signal-Fitted$. Herein there is a clear evidence that the mean value will increase for the same method from the pure to the noisy signal. This is the effect of the noise contamination which will worsen the performance of the method. The noise impact is also noted in the variance of $\Delta$ where the variability increases with a small factor in the noisy signal case. 

In case of method comparison, root-mean-square-error (RMSE) has been chosen as a good indicator. In the figure \ref{Nadya4} HSVD curve on average tends to stay above the other models. Thereby indicating a higher bias incorporated into the estimation. On the other hand, $HTLS_{_}$$PK_{_}$$FD$ outperforms HTLS based on the same graph in figure \ref{Nadya4}. Herein the RMSE tends to oscillate significantly for high level of noise and stabilizes the signal level significantly higher. However, on average the error introduced by $HTLS_{_}$$PK_{_}$$FD$ sits on lower values as compared to the HTLS case and HSVD.

The residue signal for each estimation is plotted in the figure \ref{Nadya5}, whereas the individual component superimposed together for the modeled signal are in the section \ref{Nadya6}. The estimated parameter including  amplitude, frequency, phase and damping via respective method are listed in the section \ref{Nadya7}.

\begin{figure}[!htbp]
\minipage{0.5\textwidth}%
\centering
\includegraphics[width=1\textwidth]{114.jpg}
\subcaption{Filtered Signal}\label{Nadya1}
\endminipage\hfill
\minipage{0.5\textwidth}%
\centering
\includegraphics[width=1\textwidth]{115.jpg}\\
\subcaption{Noisy Signal}\label{Nadya2}
\endminipage\hfill
\caption{Reconstruction Outcome}
\end{figure}


\begin{figure}[!htbp]
\centering
\includegraphics[width=0.7\textwidth]{RMSE.jpg}
\caption{Reconstruction Outcome}\label{Nadya4}
\end{figure}


 \begin{table}[!htbp]
\centering
\caption{Fitted performance}\label{Nadya3}

\begin{tabular}{c c c c c c c} 
\hline 

\input{Textfiles/5.txt}  

\hline 
\end{tabular}
\end{table}
 

\subsection{Subspace methods for parameter estimation}

\begin{itemize}
 
 \item In subspace method it is very hard to incorporate prior knowledge of the parameters to be estimated. However, in \cite{7} and \cite{8} the algorithm is proposed as an extension of HTLS capable of accommodating prior knowledge in the computation. It will increase the accuracy together with the resolution. Whereby it will facilitate the estimation of closed spaced peaks in the NMR spectrum with values very close to the Cramer-Rao (CR) lower bounds. The reason why the CR bounds became smaller is due to the non-zero mutual uncertainty of the parameters to be estimated. A further investigation is introduced in \ref{A2}.
 
 \item Subspace methods on the other hand are very robust framework. On other word it is a well-posed problem solution. It means that the solution exist and it is unique, nevertheless the solution is very sensitive to the initial conditions \cite{9}.
 
 \item Additionally the subspace methods are not iterative methods as compared to other optimisation based existing methods.Herein there is no cost function to be iteratively minimized, consequently the reproduction of data is ensured\cite{10}. The time complexity is much lower compare to optimisation based methods.
 
 \item Subspace based methods, however, don't suffer from local minimum. This is ensured from the lease square method wherein the best solution is outcomed. 

\item The user capability to initialize the parameters of the subspace methods is very important. In case of MRS signal the number of peaks are visually estimated. This restricts the method from being fully automatized. 

\item Furthermore, subspace methods are capable for parameter estimation of multiple different signals at the same time via the extension of the existing HTLS based algorithms\cite{10}. Consequently big data could be processed simultaneously instead of an iteration over single signal in the optimization case.


\end{itemize}



\newpage









\section{Noise removal}

Apart from artifact suppression, subspace methods can be further employed for the additional enhancement of the signal. The signal represented into a (\textbf{H}) Hankel data matrix could be seen as superposition of the $\hat{H}$ exact signal and the ($W$) pure noise. Taking into account that \textbf{wgn} is fully orthogonal to the signal to be estimated. Using an analogy of multivariate version of Pythagorean lemma for triangle, it is possible to recover the clean signal. Whereby the exact noise could be recovered via SVD as follow:


\begin{figure}[!htbp]
\minipage{0.5\textwidth}%
\centering
\begin{equation}
 \hat{H}=\hat{U}\hat{\Sigma}\hat{V^{h}}=[\hat{U_{1}} \hat{U_{2}}]
\begin{pmatrix}
 \hat{\Sigma_{1}}&0\\
 0&0
 \end{pmatrix}
 [\hat{V_{1}} \hat{V_{2}}]
 \end{equation}
\endminipage\hfill
\minipage{0.5\textwidth}%
\centering
 \begin{equation}\label{eq3}
 H=U \Sigma V^{h}=[U_{1} U_{2}]
\begin{pmatrix}
 \Sigma_{1}&0\\
 0&\Sigma_{2}
 \end{pmatrix}
 [V_{1} V_{2}] 
\end{equation}
\endminipage\hfill
\end{figure}

This estimation is possible under the fulfillment of the following conditions:
\begin{itemize}
    \item the clean signal is orthogonal to the noise $\hat{H^{T}}W=0$
    \item orthogonality of matrices $\hat{V_{1}}$ and $\hat{V_{1}}$ must be preserved in the inner product $\hat{V_{1}}W^{T}W\hat{V_{2}}=0$ \footnote{$W^{T}W=\lambda I$} 
    \item the smallest singular values $\sigma_{k}$ ins $\Sigma_{1}$ have to be larger than the largest noise singular value $\sigma_{K+1}$ in the $\Sigma_{2}$    
\end{itemize}

Noise separation performance depends critically on the the three criteria above, which in practice are never achieved, apart for the last condition. In order to satisfy the first criteria as the most important one it could be seen as an optimisation method. Minimizing the cost function in equation \ref{eq2} is completed by computing the transformation matrix $T$ such as:

\begin{equation}\label{eq2}
min||HT-\hat{H}||^{2}\rightarrow T=(H^{T}H)^{-1}H^{T}\hat{H}\rightarrow HT=H(H^{T}H)^{-1}H^{T}\hat{H}
\end{equation}

This estimation is known to be the minimum variance (MV) \cite{11} which gives the orthogonal projection $\hat{H}$ into the column space of $H$. Combining equations \ref{eq3} and \ref{eq2} the final estimation is equated as below:

\begin{equation}\label{eq5}
HT=UU^{T}\hat{H}=U_{1}U_{1}^{T}\hat{U_{1}}\hat{\Sigma_{1}}\hat{V_{1}^{T}}=U_{1}(\Sigma^{2}_{1}-\sigma^{2}_{w}I_{K})\Sigma_{1}^{-1}V_{1}^{H}=U_{1}(\Sigma^{2}_{1}-L\sigma^{2}_{v}I_{K})\Sigma_{1}^{-1}V_{1}^{H}
\end{equation}

Herein can be seen that the system is introducing a bias in the estimation as far as $U_{1}\neq\hat{U_{1}}$ and $\sigma^{2}_{v}=\frac{1}{L(M-K)}\sum_{i=K+1}^{M}\sigma_{1}^{2}$ is the average value of the noisy singular values. An outline of this algorithm could also be found in \ref{Ap3}. This method is a further improvement of Cadzow algorithm \cite{5} where the estimation is performed as $HT=U_{1}(\Sigma_{1})V_{1}^{H}$.

Apart from single channel method, multichannel Cadzow $MCC$ is also executed in here for comparative studies, where the NMR signals from different voxel locations are utilized \cite{14}. The voxels are the neighbours of the voxel of interest (VOI) which are located around into a horizontal cross-section. The method is also outlined in the section \ref{Ap31} adapted from \cite{13} and \cite{14}. This multichannel algorithm could further be enhanced (Multi-Channel Cadzow Enhanced (MCCE)) by taking into consideration different model order as well as different neighbouring size (3x3,5x5,7x7). In addition, different neighbour method is also investigated counting here square, diagonal and antidiagonal positioning. These four algorithm are executed on the data samples from NMR signal acquired in a human brain for 1024 voxels (32x32). 

In order to quantify the performance of each algorithm the signal-to-noise level is computed via equation:
\begin{equation}\label{SNR1}
SNR=20*\log\bigg\{\frac{var(Signal_{Filtered})}{var(Noise_{Residue})}\bigg\}
\end{equation} 
$Noise_{Residue}$ is the last 200 (arbitrary) sample of the filtered signal since in the FID case the main signal is at the beginning of the induction. Whereas the oscillation in the signal happening at the very end are considered noise since they do not came from any activity in the brain. 





\begin{figure}[!htbp]
\minipage{0.5\textwidth}%
\centering
\includegraphics[width=1\textwidth]{final.jpg}
\subcaption{Reconstructed Signal}\label{Nadya12}
\endminipage\hfill
\minipage{0.5\textwidth}%
\centering
\includegraphics[width=1\textwidth]{final1.jpg}
\subcaption{Filtered Signal}\label{Nadya11}
\endminipage\hfill
\caption{Final results. }
\end{figure}


After removing the water component from all the signals at different voxels as in section \ref{sec1} the enhancement of the signal is performed via Cadzow, MV, MCC and MCCE and the results are outlined in figure \ref{Nadya11}. Furthermore, the parameter estimation for respective signal is performed using HTLS method as in the section \ref{sec2} and the table of values are listed in section \ref{Ap32}. Via the parameter estimated for each algorithm the reconstruction signal is obtained. The result of the reconstructed signal are drafted in figure \ref{Nadya12}. 





 \begin{table}[!htbp]
\centering
\caption{SNR evaluation for different methods}\label{tab11}

\begin{tabular}{c c c c c c} 
\hline 
$ $&$Signal$&$Cadzow$&$MV$&$MCC$&$MCCE$ \\
\hline 
$SNR$&$ 31.3230$&$ 61.2315$&$ 64.9762$&$ 68.9809$&$ 98.9809$\\
\hline 
\end{tabular}
\end{table}

As it is expected the SNR value increases after applying Cadzow as the first enhancement algorithm. The SNR is higher case of MV as compareD to Cadzow since the filtering of noise is higher due to the higher reduction of the principal singular values in equation \ref{eq5}. Furthermore, the multichannel algorithm yields better performance as compare to the both Cadzow and MV since it takes into account 3x3 neighbours. Overall, the MCCE outperforms the previous methods since it has been optimized over the three parameter models, model order, neighbouring distance and manner. The MCCE is indicated to have the best performance for model order of 26 over a neighborhood of 7x7 for in an square orthogonal manner.


\subsection{Cadzow vs Minimum Variance}

These methods are intrinsically quite similar to each other with the main difference that instead of computing the LS solution for the signal it does compute the minimum variance MV equation \ref{eq5}. Whereby the first criteria is better approximated with MV. In this way the separation of noise outperforms Cadzow algorithm since the orthogonality is much higher in the MV case thus the introduced bias is also lower \cite{11}. The SNR increases slightly for the MV case listed in the table \ref{tab11} and it provides much better results for the parameter estimation step. The closely spaced peaks are easier to be estimated via HTLS in case of the high SNR for the signal. Since the gap between the signal and the noise singular values is smaller in the MV case, the estimated parameters preserve better accuracy \cite{11} 

\subsection{Cadzow vs Multi-Channel Cadzow}

The very main difference between Cadzow and Multichannel Cadzow is that correlation between the channels can be fully exploited thereby a significant improvement of SNR has been observed in this case\cite{14}  as it is also testified in table \ref{tab11}. This is an important feature for multichannel data which enables the estimation of the common dynamics among the systems. Common dynamic among multichannel's means the same signal interfere throughout the voxels which is superimposed at different location in the brain at different phase and amplitude. This method is also estimated to be performed quite good at low spatial correlation of the NMR signal\cite{15}.

In this specific case the signal coming from one voxel will be intervened from the other surrounding voxels. However the scale of the correlation is different depending on the type of tissue which sit on the voxel and the geometrical distance between each voxel. In the Cadzow case this is not taken into consideration therefore the signal is significantly lower. 

In this case only one of the only one of the surrounding voxels is taken into consideration, meaning that this algorithm is totally  blind from the further tissues (neighbouring voxels). Nevertheless it outperforms the traditional Cadzow subspace algorithms. 

In multichannel approach, the signal of interest is approximated via a linear combination of the information coming from all the decomposed subspaces of the signal itself together with the other common subspace information coming from the other neighbouring voxels. Therefore, combination of this information will significantly enhance the signal  and therefore the unwanted signals are suppressed quite efficiently.  


\subsection{Cadzow vs Optimized Multi-Channel Cadzow }


A further improvement of the of the MCC is evidenced. As already stated, Cadzow is trying to suppress the noise into a single channel whereas MCCE attempt to provide the best performance possible.

At first the algorithm tries to observe the difference correlation between neighbours voxels along different direction. Whereby it considers only those specific voxels which contribute the most to the common dynamics. Unlike the simplest version of Cadzow, MCCE reveals that the vertically and horizontally voxels due to their proximity to the voxel of interest share much higher mutual information regarding the metabolite. 

After decomposition of this signals into different subspaces, the reconstructed signal takes into account the number subspaces where the information is correlated mostly to the voxel of interest. Consequently big amount of consistent information is explored and therefore the enhancement is far better compare to the simple Cadzow.

If the number of neighbourhood increases this means that the number of common dynamics event is also increased. Consequently the SNR of the final signal is testified to be higher. It is also totally not taken into account for the Cadzo algorithm. However, a high number of neighbouring voxels leads to much more heterogeneous tissues compared to the voxel of interest and thus will decay the performance of the MCCE.
\section{Multi-channel data}

In this study an ultrasound (US) dataset channel data acquired from \href{www.medicalimagingcenter.be/}{MIRC} \footnote{Medical Imaging Research Center UZ Leuven} has been utilized. The goal is to remove the US artefact and filter out the undesired part of the signal. US probe a part from the noisy contaminated signal introduces some DC level as well as low frequency signal at the onset of the time course. This appear in the US image as noise and limit the usability of this method in the near field. Additionally \textbf{\textit{wgn}} is also superimposed in the signal which needs to be removed. The US probe being used has a bandwidth from 2-7 MHz meaning that anything appearing outside this bad is considered to be noise or artefact. In this application EDS is the underlying model to be employed for subspace signal processing. Since the artefact component appear at lower frequencies with significantly higher level compare to the component sitting at the desired bad (DB) (2-7 MHz) figure \ref{US1} it is critically important to remove this component without affecting the rest of the signal. This approach would enable a much better noise suppression. 

Hereby the subspace framework is utilized wherein 28 component are computed. Then the component where their respective frequency sits outside the DB will be deduced from the original signal. The outcome from this is quite on an acceptable range, since the component of the DB sits at high values compare to the rest of the components \ref{US2}. The original and the artefact removed signal in time domain are plotted in figure \ref{UST1} and \ref{UST2} respectively.


\begin{figure}[!htbp]
\minipage{0.5\textwidth}%
\centering
\includegraphics[width=1\textwidth]{800.jpg}
\subcaption{Original signal}\label{US1}
\endminipage\hfill
\minipage{0.5\textwidth}%
\centering
\includegraphics[width=1\textwidth]{801.jpg}
\subcaption{Artefact removed signal}\label{US2}
\endminipage\hfill
\caption{US signals.}
\end{figure}

After completing the artefact removal, signal has to be filtered out via subspace signal processing methods. Single channel algorithm Cadzow and minimum variance are alternative solution however their performance would not be the same as compare to the multichannel version of Cadzow. Single channel would not get into account the common dynamics of the of neighbouring region of interest being scanned as well as the cross talk between neighbouring ultrasound transducers inside the probe. The only drawback associated with multichannel is its time complexity increases significantly as the number of data-points is much higher. Nevertheless this could be a powerful approach for imaging stationary data via echo whereby an increase in image resolution could be achieved. 
 




\begin{figure}[!htbp]
\centering
\includegraphics[width=1\textwidth]{802.jpg}
\caption{US filtered signal}\label{US3}
\end{figure}


Since subspace signal processing is a parametric method, component number of the decomposition is the only input required for this underlying model. In this case, 28 components are chosen to be sufficient number since a higher value would not yield significantly better outcome, given the time complexity would increases dramatically. 

Multichannel version would thereby overcome both this obstacles and the outperform both Cadzow and minimum variance. The filtered signal in frequency domain are overlapped in figure \ref{US3} where their respective time domain time course could be found in figure \ref{UST3},\ref{UST4},\ref{UST5}.In figure \ref{US3} are the normalised spectrum where at first glance the the filtering method suppresses the undesired spectrum down to -20 dB whereby the DB is totally untouched. 

\begin{table}[!htbp]
\centering
\caption{SNR evaluation for different methods}\label{tab12}
\begin{tabular}{c c c c c c} 
\hline 
$ $&$Original Signal$&$Cadzow$&$MV$&$MCC$ \\\hline 
            
$SNR$&$ -5.2161 $&$ 10.3109 $&$ 10.3109  $&$ 10.7620$\\
\hline 
\end{tabular}
\end{table}

In order to testify the performance of these filtering method the SNR has been computed via equation \ref{SNR1} in the frequency domain. The numerator is the variance of the DB whereas the denominator is the variance of all the components outside the DB. In table \ref{tab12} are the SNR listed where it could be increase of signal quality where the multichannel case outperforms the rest of the methods. Quite important to mention that high dynamics of the signal (very high frequency components) being filtered out via multichannel is visually notable in figure \ref{US3}.




\clearpage

\bibliographystyle{unsrt}
\bibliography{Bibl}

\clearpage
\appendix

\section{Additional figure}\label{S1A1}

\subsection{ Outcome for different model order}\label{A1}




\begin{figure}[!htbp]
\foreach \i in {3,...,10} {%
    \begin{subfigure}[p]{0.5\textwidth}
        \includegraphics[width=0.85\linewidth]{\i}
    \end{subfigure}\quad
}
\end{figure}



\begin{figure}[!htbp]
\foreach \i in {11,...,18} {%
    \begin{subfigure}[p]{0.5\textwidth}
        \includegraphics[width=0.85\linewidth]{\i}
    \end{subfigure}\quad
}
\end{figure}


\begin{figure}[!htbp]
\foreach \i in {19,...,26} {%
    \begin{subfigure}[p]{0.5\textwidth}
        \includegraphics[width=0.85\linewidth]{\i}
    \end{subfigure}\quad
}
\end{figure}


\begin{figure}[!htbp]
\foreach \i in {27,...,28} {%
    \begin{subfigure}[p]{0.5\textwidth}
        \includegraphics[width=0.85\linewidth]{\i}
    \end{subfigure}\quad
}
\end{figure}


\begin{figure}[!htbp]
\centering
\includegraphics[width=0.5\textwidth]{icon3.png}\\
\caption{Exponentially damping sinusoidal underlying model}\label{EDS}
\end{figure}

\section{Parameter estimation}

\subsection{Residue signal after reconstruction}\label{Nadya5}



\begin{figure}[!htbp]
\minipage{0.5\textwidth}%
\centering
\includegraphics[width=1\textwidth]{103.jpg}\\
\includegraphics[width=1\textwidth]{105.jpg}\\
\includegraphics[width=1\textwidth]{107.jpg}\\
\subcaption{Filtered Signal}
\endminipage\hfill
\minipage{0.5\textwidth}%
\centering
\includegraphics[width=1\textwidth]{109.jpg}\\
\includegraphics[width=1\textwidth]{111.jpg}\\
\includegraphics[width=1\textwidth]{113.jpg}\\
\subcaption{Noisy Signal}
\endminipage\hfill
\caption{Residue Signal}
\end{figure}

\newpage
\subsection{ Individual component for each method} \label{Nadya6}
\newcounter{themenumber}
\forloop[2]{themenumber}{102}{\value{themenumber} < 113}{

    %\arabic{themenumber}
    \begin{figure}[!htbp]
    \centering
    \includegraphics[width=1\textwidth]{\arabic{themenumber}.jpg}
    \caption{Individual component}
    \end{figure}
}



\newpage
\subsection{Parameter estimation}\label{Nadya7}

 \begin{table}[!htbp]
\centering
\caption{Frequency estimation \textbf{\textit{Hz}}}

\begin{tabular}{c c c c c c c c c c c c c c c c c c c c c c c c c c c c c c c } 
\hline 

\input{Textfiles/11.txt}  
\input{Textfiles/1.txt}  

\hline 
\end{tabular}
\end{table}





 \begin{table}[!htbp]
\centering
\caption{Damping estimation \textbf{\textit{Hz}}}

\begin{tabular}{c c c c c c c c c c c c c c c c c c c c c c c c c c c c c c c } 
\hline 
\input{Textfiles/11.txt} 
\input{Textfiles/2.txt}


\hline 
\end{tabular}
\end{table}





 \begin{table}[!htbp]
\centering
\caption{Amplitude estimation \textbf{\textit{a.u}}}

\begin{tabular}{c c c c c c c c c c c c c c c c c c c c c c c c c c c c c c c } 
\hline 

\input{Textfiles/11.txt} 
\input{Textfiles/3.txt}
\hline 
\end{tabular}
\end{table}



 \begin{table}[!htbp]
\centering
\caption{Phase estimation \textbf{\textit{Deg}}}

\begin{tabular}{c c c c c c c c c c c c c c c c c c c c c c c c c c c c c c c } 
\hline 
\input{Textfiles/11.txt}   
\input{Textfiles/4.txt}

\hline 
\end{tabular}
\end{table}  
    
    



\newpage

\subsection{Number of parameters}\label{A2}

Suppose we have the estimation model to be estimated with wgn\footnote{White Gaussian Noise} contamination:

\begin{equation}
x[n]=A_{0}+A_{1}n+wgn=\sum_{K=0}An^{k}+wgn
\end{equation}

We construct the Fisher information matrix in order to define the CR bound for $A_{1}$ and $A_{2}$.

\begin{equation}
    p(\hat{x};\hat{A})=\frac{1}{(2\pi\sigma^2)^{\frac{N}{2}}}exp\bigg\{-\frac{1}{2*\sigma^2}\sum_{n}(x[n]-\sum_{K=0}An^{k})^2\bigg\}
\end{equation}

\begin{equation}
    log(p(\hat{x};\hat{A}))=-\frac{N}{2}log(2\pi\sigma^2)-\bigg\{\frac{1}{2*\sigma^2}\sum_{n}(x[n]-\sum_{K=0}An^{k})^2\bigg\}
\end{equation}

\begin{equation}
   \frac{\delta log(p(\hat{x};\hat{A}))}{\delta \hat{A_{i}}} =\frac{1}{2\sigma^2}\sum_{n}2(x[n]-\sum An^{k})\frac{\delta \sum An}{A_{j}}=\frac{1}{\sigma^2}\sum_{n}(x[n]-\sum An^{k})n^{i}
\end{equation}

\begin{equation}
   \frac{\delta^2 log(p(\hat{x};\hat{A}))}{\delta \hat{A_{i}}\hat{A_{j}}} =\frac{1}{\sigma^2}\sum_{n}(n^{i}-\frac{\delta\sum An^{k}}{\delta A_{j}})=\frac{-1}{\sigma^2}\sum_{n}n^{i}n^{j}\Longrightarrow I(\hat{A})=I_{ij}=\frac{1}{\sigma^2}\sum_{n}n^{i}n^{j}
\end{equation}

Thereby it is concluded that information matrix $I(A_{0}A_{1})$ is :\\

$I = 
 \begin{pmatrix}
  \sum_n n^{0}n^{0} &  \sum_n n^{0}n^{01} \\
  \sum_n n^{1}n^{0} &  \sum_n n^{1}n^{1} 
 \end{pmatrix}
 = 
 \begin{pmatrix}
  N&  N(N+1)/2 \\
  N(N+1)/2 &  N(N+1)(2N+1)/6 
 \end{pmatrix}$


In case $\sigma=1,N=10$:\\

$I = 
 \begin{pmatrix}
 10&55 \\
  55&385 
 \end{pmatrix}$
 
 Since the CRLB\footnote{CRLB=Cramer-Rao Lower Bound} is the inverse of the information  matrix
$CRLB=
 \begin{pmatrix}
  0.4667&-0.0667 \\
  -0.0667&0.121 
 \end{pmatrix}$

Therefore
\begin{equation}
Var(\hat{A_{0}})\geq 0.4667=CRLB_{\hat{A_{0}}}   
\end{equation}
and
\begin{equation}
Var(\hat{A_{0}})\geq 0.4667=CRLB_{\hat{A_{1}}}
\end{equation}

However if $A_{1}$ is known, then 

\begin{equation}
I(A_{0})=\frac{1}{\sigma^2}[\sum_{n}n^{0}n^{0}]
\end{equation}
and for $\sigma=1,N=10$ $I(A_{0})=10$ and 

\begin{equation}
Var(\hat{A_{0}})\geq 0.4667=CRLB_{\hat{A_{0_{'}}}}
\end{equation}

Hence, knowing the value of $A_{1}$, $A_{0}$ can be estimated with less certainty. This is because $CRLB(A_{1}, A_{0})$ is not diagonal, meaning there is a non-zero mutual uncertainty in the values of $A_{1}$, $A_{0}$. Consequently, knowing $A_{1}$ decreases the uncertainty in $A_{0}$. CRLB increases consistently if the number of parameters to be estimated increases and decreases always information is provided.



\newpage
\subsection{Revoming the known part in the data}\label{Ap3}

The water filtered signal is restructured into a Hankel Matrix $LxM$ 

$X= 
 \begin{pmatrix}
  x_{0} & x_{1} & \cdots & x_{m} \\
  x_{1} & x_{2} & \cdots &   \\
  \vdots  & \vdots  & \ddots & \vdots  \\
  x_{L-1} & \vdots & \cdots & x_{N-1} 
 \end{pmatrix}$

Since the date is noise free the the matrix X has a Vandermonde decomposition of the form $X=SCT^{T}$ where


\begin{figure}[!htbp]
\minipage{0.33\textwidth}%
\centering
$S= 
 \begin{pmatrix}
  1 & 1 & \cdots & 1 \\
  Z_{1}^{1} & Z_{2}^{1} & \cdots &   Z^{1} \\
  \vdots  & \vdots  & \ddots & \vdots  \\
  Z_{1}^{L-1} & \vdots & \cdots & Z^{L-1} 
 \end{pmatrix}$
\endminipage\hfill
\minipage{0.33\textwidth}%
\centering
 $C= 
 \begin{pmatrix}
  c_{1} & 0 & \cdots & 0 \\
  0 & c_{2}& \cdots &   0 \\
  \vdots  & \vdots  & \ddots & \vdots  \\
  0 & \vdots & \cdots & c 
 \end{pmatrix}$
\endminipage\hfill
\minipage{0.33\textwidth}%
\centering
$T= 
 \begin{pmatrix}
  1 & 1 & \cdots & 1 \\
  Z_{1}^{1} & Z_{2}^{1} & \cdots &   Z^{1} \\
  \vdots  & \vdots  & \ddots & \vdots  \\
  Z_{1}^{M-1} & \vdots & \cdots & Z^{M-1} 
 \end{pmatrix}^{T}$
\endminipage\hfill
\end{figure}
 where K is the model order to be estimated $L\geq k,M\geq K, N=L+M-1$.
 
 The p known poles are then indexed as the first one $z,K=1\cdots ,p$ and the rest $z, k=p+1\cdots ,K$ are the unknown poles. Therefore the first p columns $S_{p}$ and $T_{p}$ of the of the matrixes $S$ and $T$ are know priory. After QR decomposition of the $T_{p}$ the data are projected onto the orthogonal subspace.
 
 \begin{figure}[!htbp]
\minipage{0.33\textwidth}%
\centering
 \begin{equation}
 T_{p}=[Q_{1} Q_{2}][R^{T} 0]^{T}
 \end{equation}
\endminipage\hfill
\minipage{0.33\textwidth}%
\centering
 \begin{equation}
 \hat{X}=XQ_{2}^{*}=S_{K-p}C_{K-p}T_{K-p}^{T}Q_{2}^{*}
 \end{equation}
\endminipage\hfill
\end{figure}


 
 

 
 where $ \hat{X}$ are the newly projected data, and $Q_{2}^{*}$ is the conjugation of $Q_{2}$. 
 
 
 
\newpage
\subsection{Parameter estimation algorithm}\label{Ap2}

\textbf{Step1:}Arrange the data points $x_{n},n=1\cdots N-1$ in a $(LxM)$-Hankel matrix \textbf{H}, $N+L+M-1,L>K$ 

\begin{equation}
H= 
 \begin{pmatrix}
  x_{0} & x_{1} & \cdots & x_{m} \\
  x_{1} & x_{2} & \cdots &   \\
  \vdots  & \vdots  & \ddots & \vdots  \\
  x_{L-1} & \vdots & \cdots & x_{N-1} 
 \end{pmatrix}
\end{equation}

 
 \textbf{Step2:} Compute the SVD of \textbf{H}
 \begin{equation}
 H=U_{Lxmin(L,M)}\sum_{min(L,M)*min(L,M)}V^{H}_{Mxmin(L,M)}
 \end{equation}
 
 \textbf{Step3:} Truncate the SVD of \textbf{H} on order to outcome the best rank-K' approximation 
 
 \begin{equation}
 \hat{H}=\hat{U}_{LxK'}\hat{\sum}_{K'*K'}\hat{V}V^{H}_{MxK'}
 \end{equation}
 
 The rank K' is equal to the model order which correspond to the number of complex exponential in the signal. If the signal is real, then K' is twice the model order.
 
 \textbf{Step4:} Form the overdetermined set of equation
 
 \begin{equation}
 \hat{U\downarrow}\approx\hat{U\uparrow}\tilde{Z}
 \end{equation}
 
 where $\hat{U\downarrow}$ and $\hat{U\uparrow}$ are derived from $\hat{U}$ by omitting its firs and last row respectively:
 
 \begin{itemize}
     \item HSVD: compute an estimate of \tilde{Z} by solving hte above set of equations via LS
     \item HTLS: compute an estimate of \tilde{Z} by solving hte above set of equations via TLS
 \end{itemize}
 
The eigenvalues $\lambda$ of $\tilde{Z}$ estimated the poles of the signal $\lambda=\hat{z}=exp\{-\hat{\alpha}+2\pi j\hat{v} \delta t\}$ from where it is then very easy the estimation of the damping factor $\alpha$ and the frequencies $v$.

\textbf{Step5:} Using the estimates $\hat{z}, k=1\cdots,K$ and the signal sample $x_{n},n=0\cdots,N-1$ compute the LS solution $\hat{c_{k}}=\hat{\alpha}exp\{j\hat{\phi_{k}}\Delta t\}$

\begin{equation}
 \begin{pmatrix}
  1 & \cdots & 1 \\
  \hat{z}^{1}_{1} & \cdots &  \hat{z}^{1}_{K}  \\
  \vdots  & \ddots & \vdots  \\
    \hat{z}^{N-1}_{1} &  \cdots &   \hat{z}^{N-1}_{K} 
 \end{pmatrix}=
  \begin{pmatrix}
  c_{1}\\
  c_{1}\\
  \vdots\\
  c_{K}
   \end{pmatrix}= 
  \begin{pmatrix}
  x_{0}\\
  x_{1}\\
  \vdots\\
  x_{N-1}
   \end{pmatrix}
\end{equation}



\subsection{Parameter estimation}\label{Ap32}

 \begin{table}[!htbp]
\centering
\caption{Frequency estimation \textbf{\textit{Hz}}}

\begin{tabular}{c c c c c c c c c c c c c c c c c c c c c c c c c c c c c c c } 
\hline 

\input{Textfiles/10.txt}  
\input{Textfiles/6.txt}  

\hline 
\end{tabular}
\end{table}





 \begin{table}[!htbp]
\centering
\caption{Damping estimation \textbf{\textit{Hz}}}

\begin{tabular}{c c c c c c c c c c c c c c c c c c c c c c c c c c c c c c c } 
\hline 
\input{Textfiles/10.txt} 
\input{Textfiles/7.txt}


\hline 
\end{tabular}
\end{table}





 \begin{table}[!htbp]
\centering
\caption{Amplitude estimation \textbf{\textit{a.u}}}

\begin{tabular}{c c c c c c c c c c c c c c c c c c c c c c c c c c c c c c c } 
\hline 

\input{Textfiles/10.txt} 
\input{Textfiles/8.txt}
\hline 
\end{tabular}
\end{table}



 \begin{table}[!htbp]
\centering
\caption{Phase estimation \textbf{\textit{Deg}}}

\begin{tabular}{c c c c c c c c c c c c c c c c c c c c c c c c c c c c c c c } 
\hline 
\input{Textfiles/10.txt}   
\input{Textfiles/9.txt}

\hline 
\end{tabular}
\end{table}  
    
    


\newpage

\subsection{Signal enhancement algorithm}\label{Ap3}  


\textbf{Step1:}Arrange the data points $x_{n},n=1\cdots N-1$ in a $(LxM)$-Hankel matrix \textbf{H}, $N+L+M-1,L>K$ 

\begin{equation}
H= 
 \begin{pmatrix}
  x_{0} & x_{1} & \cdots & x_{m} \\
  x_{1} & x_{2} & \cdots &   \\
  \vdots  & \vdots  & \ddots & \vdots  \\
  x_{L-1} & \vdots & \cdots & x_{N-1} 
 \end{pmatrix}
\end{equation}

 
 \textbf{Step2:} Compute the SVD of \textbf{H}
 \begin{equation}
 H=U_{Lxmin(L,M)}\Sigma_{min(L,M)*min(L,M)}V^{H}_{Mxmin(L,M)}
 \end{equation}
 
 \textbf{Step3:} Truncate the SVD of \textbf{H} on order to outcome the best rank-K' approximation 
 
 \begin{equation}
 \hat{H}=\hat{U}_{LxK'}f\big\{\hat{\Sigma}_{K'*K'}\big\}\hat{V}^{H}_{MxK'}
 \end{equation}
 
 The rank K' is equal to the model order which correspond to the number of complex exponential in the signal. If the signal is real, then K' is twice the model order. The correction is different of the singular values is $f\big\{\hat{\Sigma}_{K'*K'}\big\}=(\Sigma_{1})$  for Cadzow and $f\big\{\hat{\Sigma}_{K'*K'}\big\}=(\Sigma^{2}_{1}-L\sigma^{2}_{w}I_{K})\Sigma_{1}^{-1}$ for MV.
 
\textbf{Step4:} Average along the anti-diagonal the the reconstructed matrix \hat{H} which leads to a the the Hankel matrix

\begin{equation}
\hat{H}_{ant-diag-ave}=Average(\hat{H})
\end{equation}

\textbf{Step5:} Extract the first column and the last row of the newly computed Hankel matrix $\hat{H}_{ant-diag-ave}$



\subsection{Multi channel signal enhancement algorithm}\label{Ap31}  


\textbf{Step1:}Arrange the data points $x^{q}_{n},n=1\cdots N-1$ in a $(LxM)$-Hankel matrix \textbf{H}, $N+L+M-1,L>K$ for all the channel data $q=1\cdots Q$ 

\begin{equation}
H_{1}= 
 \begin{pmatrix}
  x_{0} & x_{1} & \cdots & x_{m} \\
  x_{1} & x_{2} & \cdots &   \\
  \vdots  & \vdots  & \ddots & \vdots  \\
  x_{L-1} & \vdots & \cdots & x_{N-1} 
 \end{pmatrix}\cdots \cdots 
 H_{Q}= 
 \begin{pmatrix}
  x_{0} & x_{1} & \cdots & x_{m} \\
  x_{1} & x_{2} & \cdots &   \\
  \vdots  & \vdots  & \ddots & \vdots  \\
  x_{L-1} & \vdots & \cdots & x_{N-1} 
 \end{pmatrix}
\end{equation}

 \textbf{Step2:} Form a block Hankel matrix
 \begin{equation}
 H=[H_{1}|H_{2}|\cdots H_{Q}]
 \end{equation}
 
 \textbf{Step3:} Compute the SVD of \textbf{H}
 \begin{equation}
 H=U_{Lxmin(L,M)}\Sigma_{min(L,M)*min(L,M)}V^{H}_{Mxmin(L,M)}
 \end{equation}
 
 \textbf{Step4:} Truncate the SVD of \textbf{H} on order to outcome the best rank-K' approximation 
 
 \begin{equation}
 \hat{H}=\hat{U}_{LxK'}f\big\{\hat{\Sigma}_{K'*K'}\big\}\hat{V}^{H}_{MxK'}
 \end{equation}
 
 The rank K' is equal to the model order which correspond to the number of complex exponential in the signal. If the signal is real, then K' is twice the model order. The correction is different of the singular values is $f\big\{\hat{\Sigma}_{K'*K'}\big\}=(\Sigma_{1})$  for Cadzow and $f\big\{\hat{\Sigma}_{K'*K'}\big\}=(\Sigma^{2}_{1}-L\sigma^{2}_{w}I_{K})\Sigma_{1}^{-1}$ for MV.
 
 
\textbf{Step5:} Average along the anti-diagonal the the reconstructed matrix $\hat{H}$ which leads to a the the Hankel matrix

\begin{equation}
\hat{H}_{ant-diag-ave}=Average(\hat{H})
\end{equation}



 \textbf{Step5:} Extract  the the Hankel matrix corresponding to the voxel of interest $H_{interest}$
 
\textbf{Step6:} Extract the first column and the last row of the newly computed Hankel matrix $H_{interest}$
\subsection{Parameter estimation}\label{Ap4}



\begin{figure}[!htbp]
\minipage{0.5\textwidth}%
\centering
\includegraphics[width=1\textwidth]{803.jpg}
\subcaption{Original signal}\label{UST1}
\includegraphics[width=1\textwidth]{804.jpg}
\subcaption{Artefact removed signal}\label{UST2}
\endminipage\hfill
\minipage{0.5\textwidth}%
\centering
\includegraphics[width=1\textwidth]{805.jpg}
\subcaption{Cadzow filtered signal}\label{UST3}
\includegraphics[width=1\textwidth]{806.jpg}
\subcaption{Minimum variance filtered signal}\label{UST4}
\endminipage\hfill
\caption{Time domain signal}
\end{figure}

\begin{figure}[!htbp]
\centering
\includegraphics[width=1\textwidth]{807.jpg}
\caption{Multichnnale outcome}\label{UST5}
\end{figure}

%
\lstset{language=Matlab,%
    %basicstyle=\color{red},
    breaklines=true,%
    morekeywords={matlab2tikz},
    keywordstyle=\color{blue},%
    morekeywords=[2]{1}, keywordstyle=[2]{\color{black}},
    identifierstyle=\color{black},%
    stringstyle=\color{mylilas},
    commentstyle=\color{mygreen},%
    showstringspaces=false,%without this there will be a symbol in the places where there is a space
    numbers=left,%
    numberstyle={\tiny \color{black}},% size of the numbers
    numbersep=9pt, % this defines how far the numbers are from the text
    emph=[1]{for,end,break},emphstyle=[1]\color{red}, %some words to emphasise
    %emph=[2]{word1,word2}, emphstyle=[2]{style},   
    backgroundcolor=\color{backcolour},   
    commentstyle=\color{codegreen},
    %keywordstyle=\color{magenta},
    numberstyle=\tiny\color{codegray},
    stringstyle=\color{codepurple},
    basicstyle=\footnotesize,
    breakatwhitespace=false,         
    breaklines=true,                 
    captionpos=b,                    
    keepspaces=true,                 
    numbers=left,                    
    numbersep=5pt,                  
    showspaces=false,                
    showstringspaces=false,
    showtabs=false,                  
    tabsize=2
}

\newpage
\section{Artefact removal code}

\subsection{Main code}
\lstinputlisting{Vangjush_Main_Exercise_Session_2.m}
\subsection{Aid functions}

\lstinputlisting{Vangjush_AddNoise.m}
\lstinputlisting{Vangjush_Automated_CCA.m}
\lstinputlisting{Vangjush_Best_Ordering.m}
\lstinputlisting{Vangjush_BSS_Evaluate.m}
\lstinputlisting{Vangjush_BSS_Final.m}
\lstinputlisting{Vangjush_CCA.m}
\lstinputlisting{Vangjush_Centering.m}
\lstinputlisting{Vangjush_Decompose_BSS.m}
\lstinputlisting{Vangjush_Fast_ICA.m}
\lstinputlisting{Vangjush_Gaussianity.m}
\lstinputlisting{Vangjush_Histogram_Amplitudes.m}
\lstinputlisting{Vangjush_ICA.m}
\lstinputlisting{Vangjush_Inner_Product.m}
\lstinputlisting{Vangjush_Inner_Product_Second.m}
\lstinputlisting{Vangjush_Manual_CCA.m}
\lstinputlisting{Vangjush_PlotEEG.m}
\lstinputlisting{Vangjush_RMSE.m}
\lstinputlisting{Vangjush_RMSE_BSS.m}
\lstinputlisting{Vangjush_Save_Images.m}
\lstinputlisting{Vangjush_SignalProjection.m}
\lstinputlisting{Vangjush_Whitening.m}
\lstinputlisting{Vangjushu_Canonical_Coefficients.m}




\section{Nonlinear Signal Processing code}

\subsection{Main code}
\lstinputlisting{Vangjush_Main_Exercise_Session.m}
\subsection{Aid functions}
\lstinputlisting{Vangjush_ACF.m}
\lstinputlisting{Vangjush_ACF_K.m}
\lstinputlisting{Vangjush_Box_Counting.m}
\lstinputlisting{Vangjush_Compute_Fractial_Dimension.m}
\lstinputlisting{Vangjush_Correlation_Integral.m}
\lstinputlisting{Vangjush_DFA.m}
\lstinputlisting{Vangjush_F_slope.m}
\lstinputlisting{Vangjush_First_Zero.m}
\lstinputlisting{Vangjush_Grassberger_Procaccia.m}
\lstinputlisting{Vangjush_Hurst_Exponential.m}
\lstinputlisting{Vangjush_LSE_Of_Window.m}
\lstinputlisting{Vangjush_Lyapunov_Wolf.m}
\lstinputlisting{Vangjush_Matrix_Adjective.m}
\lstinputlisting{Vangjush_Nonlinear_Signal_Processing.m}
\lstinputlisting{Vangjush_Optional.m}
\lstinputlisting{Vangjush_Optional_Plots.m}
\lstinputlisting{Vangjush_Parameter_2_Latex_Table.m}
\lstinputlisting{Vangjush_Parameter_2_Matrix.m}
\lstinputlisting{Vangjush_Phase_Space_Reconstrucion.m}
\lstinputlisting{Vangjush_Poincare.m}
\lstinputlisting{Vangjush_Produce_Images.m}
\lstinputlisting{Vangjush_Random_Walk.m}
\lstinputlisting{Vangjush_Sample_Entropy.m}
\lstinputlisting{Vangjush_Save_Images.m}
\lstinputlisting{Vangjush_Windowing_Function.m}















\end{document}

